\documentclass[a4paper,12pt]{exam}
	\usepackage{graphicx}
	\usepackage[utf8]{inputenc}
	\usepackage[T1]{fontenc}
	\usepackage{listings}
	\usepackage{color}
	\usepackage{amsmath}
	\usepackage{enumerate}
	\usepackage{caption}
	\usepackage{subcaption}
	\definecolor{dkgreen}{rgb}{0,0.6,0}
	\definecolor{gray}{rgb}{0.5,0.5,0.5}
	\definecolor{mauve}{rgb}{0.58,0,0.82}

	\lstset{frame=tb,
	  language=Python,
	  aboveskip=3mm,
	  belowskip=3mm,
	  showstringspaces=false,
	  columns=flexible,
	  basicstyle={\small\ttfamily},
	  numbers=none,
	  numberstyle=\tiny\color{gray},
	  keywordstyle=\color{blue},
	  commentstyle=\color{dkgreen},
	  stringstyle=\color{mauve},
	  breaklines=true,
	  breakatwhitespace=true
	  tabsize=3
	}

\begin{document}
\begingroup 
	  \bf \Large Mecânica Clássica I\\
	  \indent \normalsize André Del Bianco Giuffrida
	\endgroup
	\\ \quad
	\\
	Um foguete acha-se em órbita elíptica em torno da Terra, perigeu $r_1$, apogeu $r_2$, medidos a partir do centro da Terra. Em certo ponto de sua órbita, o motor é ligado durante um tempo curto para fornecer um acréscimo $\Delta v$ na velocidade que coloca o foguete em órbita e que permite escapar à velocidade $v_0$ relativa à terra. Mostre que $\Delta v$ é um mínimo, se o impulso for aplicado no perigeu e paralelo a velocidade orbital. Determine $\Delta v$ para este caso, em termo dos parâmetros da orbita elíptica $\epsilon$ e $a$; a aceleração $g$; a distância $R$ do centro da Terra e a velocidade final $v_0$. Você pode explicar sob as leis da Física porque $\Delta v$ será tanto menor quanto maior for $\epsilon$?
	\\
	\\
	Sendo :
	\[\vec{v} = \dot r \hat r + r \dot\theta \hat \theta \quad \text{,} \quad r(\theta) = \frac{a(1-\epsilon^2) }{1+\epsilon cos(\theta)} \quad \text{,} \quad \dot r = \frac{a(1-\epsilon^2) sin(\theta) \dot\theta}{(1+\epsilon cos(\theta))^2}\]
	
	\end{document}
